\usepackage{tcolorbox}
\tcbuselibrary{theorems}
\tcbuselibrary{skins}
\tcbuselibrary{breakable}
\tcbset{
defstyle/.style={enhanced,
                  breakable,
                   colback=white,
                    colbacktitle=green!20,
                    coltitle=black,
                  fonttitle=\bfseries\upshape,
                  separator sign={.},
                  colframe=white,
                 leftrule=12mm,
                 overlay={\node[anchor=north west,rectangle,minimum width=1cm,minimum height=1cm,pattern={bricks},pattern color=green] at (frame.west) {}; }},
esestyle/.style={enhanced,
                  breakable,
                   colback=white,
                    colbacktitle=orange!20,
                    coltitle=black,
                  fonttitle=\bfseries\upshape,
                  separator sign={.},
                  colframe=white,
                 leftrule=12mm,
                 overlay={\node[anchor=north west,rectangle,minimum width=1cm,minimum height=1cm,pattern={sixpointed stars},pattern color=orange] at (frame.west) {}; }},
attstyle/.style={enhanced,
                  breakable,
                   colback=white,
                    colbacktitle=red!20,
                    coltitle=black,
                  fonttitle=\bfseries\upshape,
                  separator sign={.},
                  colframe=white,
                 leftrule=12mm,
                 overlay={\node[anchor=north west,rectangle,minimum width=1cm,minimum height=1cm,pattern={checkerboard},pattern color=red] at (frame.west) {}; }},        
ossstyle/.style={enhanced,
                  breakable,
                   colback=white,
                    colbacktitle=olive!20,
                    coltitle=black,
                  fonttitle=\bfseries\upshape,
                  separator sign={.},
                  colframe=white,
                 leftrule=12mm,
                  overlay={\node[anchor=north west,rectangle,minimum width=1cm,minimum height=1cm,pattern={crosshatch dots},pattern color=olive] at (frame.west) {}; }},           
procstyle/.style={enhanced,
                  breakable,
                   colback=white,
                    colbacktitle=blue!20,
                    coltitle=black,
                  fonttitle=\bfseries\upshape,
                  separator sign={.},
                  colframe=white,
                 leftrule=12mm,
                  overlay={\node[anchor=north west,rectangle,minimum width=1cm,minimum height=1cm,pattern={fivepointed stars},pattern color=blue!75!black] at (frame.west) {}; }},           
princstyle/.style={enhanced,
                  breakable,
                   colback=white,
                    colbacktitle=brown!20,
                    coltitle=black,
                  fonttitle=\bfseries\upshape,
                  separator sign={.},
                  colframe=white,
                 leftrule=12mm,
                 overlay={\node[anchor=north west,rectangle,minimum width=1cm,minimum height=1cm,pattern={north west lines},pattern color=brown] at (frame.west) {}; }}, 
teostyle/.style={enhanced,
                  breakable,
                   colback=white,
                    colbacktitle=cyan!20,
                    coltitle=black,
                  fonttitle=\bfseries\upshape,
                  separator sign={.},
                  colframe=white,
                 leftrule=12mm,
                 overlay={\node[anchor=north west,rectangle,minimum width=1cm,minimum height=1cm,pattern={grid},pattern color=cyan] at (frame.west) {}; }},                  
}
\newtcolorbox{lattention}{breakable,enhanced,arc=0mm,colback=gray!5,colframe=gray,leftrule=12mm,%
overlay={\node[anchor=north west,outer sep=2pt] at (frame.north west) {\includegraphics[width=8mm]{0022}}; }}
\newtcbtheorem [number within=chapter] {definizionet}{Definizione}{defstyle}{}
\newtcbtheorem [number within=chapter]{esempiot}{Esempio}{esestyle}{}
\newtcbtheorem [number within=chapter]{attenzionet}{Attenzione}{attstyle}{}
\newtcbtheorem [number within=chapter] {osservazionet}{Osservazione}{ossstyle}{}
\newtcbtheorem [number within=chapter] {procedurat}{Procedura}{procstyle}{}

\newtcbtheorem [number within=chapter] {principiot}{Principio}{princstyle}{}
\newtcbtheorem [number within=chapter] {teoremat}{Teorema}{teostyle}{}