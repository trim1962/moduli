\usepackage{calc}
\usepackage{ifthen}
%\DeclareTextCommand{\nobreakspace}{T1}{\leavevmode\nobreak\ }
\newcommand{\nobs}{\nobreakspace}
\newboolean{puntinisi}
\newcommand{\stampapuntini}{\setboolean{puntinisi}{true}}
\newcommand{\nonstampapuntini}{\setboolean{puntinisi}{false}}
\makeatletter
\newcommand\puntini[1]{%
\ifthenelse{\boolean{puntinisi}}{%
\ifmmode\expandafter\@firstoftwo\else
\expandafter\@secondoftwo\fi
{\makebox[\widthof{$#1$}]}{\makebox[\widthof{#1}+15pt]}{\dotfill}}{#1}}
\makeatother
% % % % % % % tasti calcolatrice
\newcommand{\tasto}[1]{\tikz[baseline=-.5ex]
	\node[fill=white,drop shadow={shadow xshift=0.5ex,shadow yshift=-0.5ex,fill=black,opacity=0.75},rectangle,draw,minimum height=.6cm,minimum width=1.2cm, rounded corners=2.0pt,inner sep=1pt,font=\sffamily] {#1};
}
\newcommand{\tastoshift}{\tasto{shift}}
\newcommand{\tastopgreco}{\tastoshift\tasto{$\pi$}}
\newcommand{\tastopiu}{\tasto{$+$}}
\newcommand{\tastomeno}{\tasto{$-$}}
\newcommand{\tastoparentesisin}{\tasto{$($}}
\newcommand{\tastoparentesides}{\tasto{$)$}}
\newcommand{\tastoquadrato}{\tasto{$x^2$}}
\newcommand{\tastoradicequadrata}{\tasto{$\sqrt{\phantom{x^2}}$}}
\newcommand{\tastoper}{\tasto{$\times$}}
\newcommand{\tastodiv}{\tasto{$\div$}}
\newcommand{\tastouguale}{\tasto{$=$}}
\newcommand{\tastotan}{\tasto{$\tan$}}
\newcommand{\tastocos}{\tasto{$\cos$}}
\newcommand{\tastosin}{\tasto{$\sin$}}
\newcommand{\tastoitan}{\tastoshift\tasto{$\tan^{-1}$}}
\newcommand{\tastoicos}{\tastoshift\tasto{$\cos^{-1}$}}
\newcommand{\tastoisin}{\tastoshift\tasto{$\sin^{-1}$}}
\newcommand{\tastoans}{\tasto{Ans}}
\newcommand{\tastomode}{\tasto{Mode}}
\newcommand{\testgradi}{\begin{center}
		\begin{tabular}{ll}
		\tastosin\tasto{90}\tastouguale& 1\\ 
	\end{tabular}
	\end{center}}
\newcommand{\testradianti}{\begin{center}
		\begin{tabular}{ll}
			\tastopgreco\tastodiv\tasto{2}\tastouguale	&\num[round-precision=6,round-mode=places]{1.570796327}  \\ 
			\tastosin\tastoans\tastouguale& 1 \\ 
		\end{tabular} 
	\end{center}}
%todo inizio
\usepackage[italian,colorinlistoftodos]{todonotes}
\newcommand{\bassapriorita}[1]{\todo[color=green!40]{#1}}
\newcommand{\mediapriorita}[1]{\todo[color=blue!40]{#1}}
\newcommand{\altapriorita}[1]{\todo[color=red!40]{#1}}
%todo fine

