%italianizziomo gli operatori

%\let\sin\relax % evitiamo un messaggio di avviso
%\DeclareMathOperator{\sin}{sin}

%\let\arcsin\relax % evitiamo un messaggio di avviso
%\DeclareMathOperator{\arcsin}{arcsin}

%\let\tan\relax % evitiamo un messaggio di avviso
%\DeclareMathOperator{\tan}{tan}

%\let\arctan\relax % evitiamo un messaggio di avviso
%\DeclareMathOperator{\arctan}{arctan}

%\let\cot\relax % evitiamo un messaggio di avviso
%\DeclareMathOperator{\cot}{cotg}
%\let\Log\relax % evitiamo un messaggio di avviso
%\DeclareMathOperator{\Log}{Log}
%\let\ln\relax % evitiamo un messaggio di avviso
%\DeclareMathOperator{\ln}{ln}
\DeclareMathOperator{\arccotg}{arccotg}
%\DeclareMathOperator{\sec}{sec}
\DeclareMathOperator{\cosec}{cosec}

\providecommand{\abs}[1]{\lvert#1\rvert}
\providecommand{\norm}[1]{\lVert#1\rVert}
%operatore derivata
\newcommand{\OpD}[1]{D\left[#1\right]} 
%\newcommand{\gradi}{\ensuremath{^\circ}}
%\newcommand{\gradi}{\ensuremath{\degree}}
%\newcommand{\minuti}{\ensuremath{\arcminute}}
\newcommand{\PA}[1]{\boldsymbol{\mathcal{#1}}}
\newcommand{\PDA}{\PA{P}}

\newcommand{\numberset}[1]{\mathbb{#1}}

\newcommand{\Ni}{\numberset{N}}
\newcommand{\Nz}{\numberset{N}^{}_{0}}
\newcommand{\Z}{\numberset{Z}}
\newcommand{\Zn}{\numberset{Z^{-}}}
\newcommand{\Zp}{\numberset{Z^{+}}}
\newcommand{\Q}{\numberset{Q}}
\newcommand{\Qp}{\numberset{Q^{+}}}
\newcommand{\Qn}{\numberset{Q^{-}}}
\newcommand{\R}{\numberset{R}}
\newcommand{\Rpos}{\numberset{R^{+}}}
\newcommand{\Rneg}{\numberset{R^{-}}}
\newcommand{\nq}{\ensuremath{\mathbb{Q}}}
\newcommand{\bld}[1]{\mbox{\boldmath $#1$}}
\newcommand{\Co}{\numberset{C}}
\newcommand{\function}[5]{%
	\begin{array}{@{}r<{{}}@{}c@{}c@{}l@{}}
		#1\colon & #2 & {}\to{}     & #3 \\
		& #4 & {}\mapsto{} & #5
\end{array}}
\newcommand{\funzione}[3]{#1\colon #2\to{}#3}
%spazi fantasma
%\newcommand{\spa}{\phantom{1}}
%\let\origcleardoublepage\cleardoublepage
%\newcommand{\clearemptydoublepage}{%
%\clearpage
%{\pagestyle{empty}\origcleardoublepage}%
%}
%\let\cleardoublepage\clearemptydoublepage
%spazio insecabile
%\newcommand{\nbs}{\nobreakspace}
%le costanti
\newcommand{\costante}{\textrm{costante}}
%\DeclareMathOperator{\uimm}{j}
% The number `e'
\providecommand*{\eu}%
{\ensuremath{\mathrm{e}}}
\providecommand*{\uimm}%
%{\ensuremath{\mathrm{j}}}
{j\mkern1mu}
			% parte immaginaria di 1: \parteimm[\dot{x}(t)]
% \DeclareMathOperator{\Re}{Re}
% \DeclareMathOperator{\Im}{Im}
% \DeclareMathOperator{\Arg}{Arg}
%coniugio
\newcommand{\conj}[2][3]{{}\mkern#1mu\overline{\mkern-#1mu#2}}
\let\misura\conj
%  \renewoperator{\Re}%
%  {\mathrm{Re}}{\nolimits}
%  \renewoperator{\Im}%
%  {\mathrm{Im}}{\nolimits}
\DeclareMathOperator{\mcd}{mcd}
\DeclareMathOperator{\mcm}{mcm}
% % % %prodotto in croce
\newcommand{\prodcroce}[4]{%
	\begin{tikzpicture}[thick]
	\def\x{2.8mm}
	\def\h{2.4mm}
	\def\dist{12mm}%1cm
	\node at (0,0) {$\displaystyle \frac{#1}{#2}$};
	\node at (\dist,0) {$\displaystyle \frac{#3}{#4}$};
	\node at (2.5*\dist,0) {$#1\cdot #4=#2\cdot #3$};
	% collegamento termini
	\draw[-stealth] (\x, \h)--(\dist-\x,-\h);
	\draw[-stealth] (\x,-\h)--(\dist-\x, \h);
	\end{tikzpicture}%
}
\newcommand*{\rectangolo}[1]{\tikz[baseline=(char.base)]{
		\node[shape=rectangle,draw,inner sep=2pt] (char) {#1};}}
\newcommand*{\circled}[1]{\tikz[baseline=(char.base)]{
		\node[shape=circle,draw,inner sep=2pt] (char) {#1};}}%numeri in circonferenze
\newcommand*{\numcircledmod}[1]{\raisebox{.5pt}{\textcircled{\raisebox{-.9pt} {#1}}}}%numeri in circonferenze 2 versione 
\newcommand*{\csqrt}[2]{\sqrt[\numcircledmod{#1}]{#2}}
\DeclareMathSymbol{\vs}{\mathpunct}{letters}{"3B}
\newcommand{\coord}[2]{\left(#1\vs #2\right)}